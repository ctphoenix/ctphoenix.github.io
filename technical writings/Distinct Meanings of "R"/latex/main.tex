\documentclass{article}
\usepackage[utf8]{inputenc}
\setlength\parindent{0pt}
\usepackage{nopageno}
\usepackage{amsmath}
\usepackage{amsfonts}

\title{Two Distinct ``R''s}
\author{}
\date{}

\usepackage{natbib}
\usepackage{graphicx}
\usepackage{amsmath}
\begin{document}

\maketitle

Correlation, often denoted by "R", actually shares two definitions that are logically separable. This snippet demonstrates that.

Let $\mathbf{y}$ be a true outcome vector and $\widehat{\mathbf{y}}$ be a vector of predictions of $\mathbf{y}$. Define the (centered) sum of products $\text{SP}(\mathbf{z}, \mathbf{w}) := (\mathbf{z}-\overline{\mathbf{z}})^\top(\mathbf{w}-\overline{\mathbf{w}})$, and define the curried function $\text{SP}(\mathbf{z}) := \text{SP}(\mathbf{z}, \mathbf{z})$.

\ \\

Consider two estimands. First, the \emph{coefficient of determination}\footnote{The second equality in (\ref{eqn:coef}) follows if the estimator is unbiased: $\mathbb{E}(\widehat{y}) = \mathbb{E}(y)$}:

\begin{align}
    R^2 := 1 - \frac{\text{SP}(\mathbf{y}-\widehat{\mathbf{y}})}{\text{SP}(\mathbf{y})} = \frac{\text{SP}(\widehat{\mathbf{y}})}{\text{SP}(\mathbf{y})} \label{eqn:coef}
\end{align}

Second, the \emph{Pearson correlation} between outcomes and prediction:

\begin{align}
    \widehat{\rho} := \frac{\text{SP}(\mathbf{y},\  \widehat{\mathbf{y}})}{\sqrt{\text{SP}(\mathbf{y})\text{SP}(\widehat{\mathbf{y}})}}
\end{align}


To summarize, $\widehat{\rho}$ uses both quantities from $R^2$, in addition to a third term that does not strongly depend on the other two\footnote{The condition for equality is $\text{SP}(\mathbf{y},\  \widehat{\mathbf{y}}) = \text{SP}^{3/2}(\widehat{\mathbf{y}})\text{SP}^{-1/2}(\mathbf{y})$. \\This does not seem guaranteed, even approximately.}. So the two estimands are logically separable.

\end{document}
